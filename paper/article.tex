% interacttfssample.tex
% v1.05 - August 2017

\documentclass[]{interact}

\usepackage{epstopdf}% To incorporate .eps illustrations using PDFLaTeX, etc.
\usepackage[caption=false]{subfig}% Support for small, `sub' figures and tables
%\usepackage[nolists,tablesfirst]{endfloat}% To `separate' figures and tables from text if required

%\usepackage[doublespacing]{setspace}% To produce a `double spaced' document if required
%\setlength\parindent{24pt}% To increase paragraph indentation when line spacing is doubled
%\setlength\bibindent{2em}% To increase hanging indent in bibliography when line spacing is doubled

\usepackage[numbers,sort&compress]{natbib}% Citation support using natbib.sty
\bibpunct[, ]{[}{]}{,}{n}{,}{,}% Citation support using natbib.sty
\renewcommand\bibfont{\fontsize{10}{12}\selectfont}% Bibliography support using natbib.sty

\theoremstyle{plain}% Theorem-like structures provided by amsthm.sty
\newtheorem{theorem}{Theorem}[section]
\newtheorem{lemma}[theorem]{Lemma}
\newtheorem{corollary}[theorem]{Corollary}
\newtheorem{proposition}[theorem]{Proposition}

\theoremstyle{definition}
\newtheorem{definition}[theorem]{Definition}
\newtheorem{example}[theorem]{Example}

\theoremstyle{remark}
\newtheorem{remark}{Remark}
\newtheorem{notation}{Notation}

\begin{document}

\articletype{ARTICLE TEMPLATE}% Specify the article type or omit as appropriate

\title{Towards understanding the central limit theorem by learning Python basics}

\author{
\name{Zolt\'an Kov\'acs\textsuperscript{a}\thanks{CONTACT Z. Kov\'acs. Email: zoltan@geogebra.org}
and Alexander Thaller\textsuperscript{b}}
\affil{\textsuperscript{a}
The Private University College of Education of the Diocese of Linz,
Salesianumweg 3, Linz, Austria;
\textsuperscript{b}Linz School of Education,
Altenberger Stra\ss e 54, Linz, Austria}
}

\maketitle

\begin{abstract}

We report on a first experiment about an email based course that connects learning
Python basics and introductory probability theory. In the experiment 7 short sequences
of homework were sent out to prospective mathematics teachers who did not have
any programming background formerly, but already had some minor knowledge on probability theory.
The experiment was about to decide if learning basics of programming can promote
understanding main concepts of probability theory.
\end{abstract}

\begin{keywords}
Python; programming; probability theory; central limit theorem
\end{keywords}

\section{Introduction}

We, the authors, have been spending a reasonable time with private tutoring in many fields
of mathematics---we are teachers. And, we agree on that probability theory is maybe
the most difficult field of mathematics---if it is about \textit{understanding}. For us, rough explanation
of statistical relationships seems to be easier than explaining the delicate issues
of properties of binomial coefficients, infinite sums and convergence. Indeed, in some
sense, statistics is much easier to explain---one needs to run virtual experiments
in a web browser % add links to Steve Phelps' https://www.geogebra.org/m/bxytp4hq, Ice Cream
% https://www.geogebra.org/m/qsjbqjfe, Broken Stick
% https://www.geogebra.org/m/bu5sxbn2, Fair Coin
% + Andreas Lindner's collections: https://www.geogebra.org/m/AytaSakt, https://www.geogebra.org/m/qXPnyCTZ
and this is well supported for several years, among others, in the online version of GeoGebra.

Connecting pure mathematics and real life or virtual experiments, can be, to our experience,
extremely hard. Probability is an important item of the curriculum at secondary level,
thus it is also an important item in the teacher training. But, to our experience,
most prospective teachers \textit{never} understand the basics beyond introductory level.
Our conjecture is that the main problem lies in students' verification if theory indeed meets practice,
so, at the end of the day, a very uncertain knowledge can be observed in the students' perception. As a consequence,
in most cases, not just yesterday's and today's mathematics teachers cannot connect
theory and practice, but their students either. Roughly speaking: everybody talks about probability and
statistics, but nobody understands anything behind the scenes.

In our paper we propose a radical way, or at least, to extend the classic way how
basics of elementary probability theory should be taught. It is \textit{programming}.
Our conjecture is that understanding will be significantly improved if the main concepts
of probability theory are supported with simple computer programs. The ouput for their flexible input
should be immediately checked, eventually in a web browser, and by changing the input parameters
to a higher number, the results can be quickly generalized. We emphasize that computers
can take on the high number of computatations from human---and this is what we exactly want.
A quite simple \textit{sample space} can contain \textit{a lot} of elements, even millions or more, and it cannot
be expected that non-experts can count their elements without deeper knowledge. In fact,
experts in probability are usually experts in combinatorics as well. To reach a reasonable
level of understanding of the main concepts in probability, by using just paper and pencil,
one needs to have a very strong and \textit{safe} background in combinatorics.

In this paper we demonstrate our proposal by explaining an email based course, sent out
to 4 prospective mathematics teachers, allowing them an arbitrary time to solve the problems being set.
The students claimed that they did not have any deeper knowledge in programming formerly, but all of
them already studied probability theory at university level. Their knowledge, however, was
not yet checked via examination. The course consisted of 7 sets of homework assignments
and took place between November 2020 and January 2021, at the Johannes Kepler University of Linz, Austria.

\section{Preparations}

\section*{Acknowledgement(s)}

\begin{thebibliography}{99}


\end{thebibliography}

\end{document}
