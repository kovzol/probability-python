\documentclass{article}
\usepackage{listingsutf8,newtxtt,xcolor,graphicx}
\usepackage[hyphens]{url}
\usepackage{hyperref}
\begin{document}
\title{Übungsblatt 3}
\date{}
\maketitle
\lstset{language=Python,
  numbers=left,
  basicstyle=\ttfamily,
  keywordstyle=\bfseries\color{magenta},
  stringstyle=\color{red},
  numberstyle=\color{blue},
  breaklines=true,
  showstringspaces=false,
  backgroundcolor=\color{yellow!30},
  ndkeywordstyle=\color{magenta!70},
  commentstyle=\color{green},
  identifierstyle=\color{black},
  literate=
  {á}{{\'a}}1 {é}{{\'e}}1 {í}{{\'i}}1 {ó}{{\'o}}1 {ú}{{\'u}}1
  {Á}{{\'A}}1 {É}{{\'E}}1 {Í}{{\'I}}1 {Ó}{{\'O}}1 {Ú}{{\'U}}1
  {à}{{\`a}}1 {è}{{\`e}}1 {ì}{{\`i}}1 {ò}{{\`o}}1 {ù}{{\`u}}1
  {À}{{\`A}}1 {È}{{\'E}}1 {Ì}{{\`I}}1 {Ò}{{\`O}}1 {Ù}{{\`U}}1
  {ä}{{\"a}}1 {ë}{{\"e}}1 {ï}{{\"i}}1 {ö}{{\"o}}1 {ü}{{\"u}}1
  {Ä}{{\"A}}1 {Ë}{{\"E}}1 {Ï}{{\"I}}1 {Ö}{{\"O}}1 {Ü}{{\"U}}1
  {â}{{\^a}}1 {ê}{{\^e}}1 {î}{{\^i}}1 {ô}{{\^o}}1 {û}{{\^u}}1
  {Â}{{\^A}}1 {Ê}{{\^E}}1 {Î}{{\^I}}1 {Ô}{{\^O}}1 {Û}{{\^U}}1
  {Ã}{{\~A}}1 {ã}{{\~a}}1 {Õ}{{\~O}}1 {õ}{{\~o}}1
  {œ}{{\oe}}1 {Œ}{{\OE}}1 {æ}{{\ae}}1 {Æ}{{\AE}}1 {ß}{{\ss}}1
  {ű}{{\H{u}}}1 {Ű}{{\H{U}}}1 {ő}{{\H{o}}}1 {Ő}{{\H{O}}}1
  {ç}{{\c c}}1 {Ç}{{\c C}}1 {ø}{{\o}}1 {å}{{\r a}}1 {Å}{{\r A}}1
  {€}{{\euro}}1 {£}{{\pounds}}1 {«}{{\guillemotleft}}1
  {»}{{\guillemotright}}1 {ñ}{{\~n}}1 {Ñ}{{\~N}}1 {¿}{{?`}}1
  {Ω}{{$\Omega$}}1,
  columns=fullflexible,
  keepspaces=true
}
\makeatletter
\def\lst@outputspace{{\ifx\lst@bkgcolor\empty\color{white}\else\lst@bkgcolor\fi\lst@visiblespace}}
\makeatother
\thispagestyle{empty}
\begin{enumerate}
\item Betrachten wir folgendes Python3 Programm:
\lstinputlisting[language=Python]{../../progs/de_AT/3.py}

Erklären Sie die Bedeutung für jede Zeile des Programms in jeweils einem Satz. Hinweis:
\texttt{itertools} ist eine Sammlung von nützlichen Python-Befehlen. Dies ist notwendig
für die 5.~Zeile.

\item Nutzen Sie \texttt{list} statt \texttt{set} in der 5.~Zeile. Wie ändert sich das Resultat?

\item Zählen Sie die \texttt{Augenzahlen} in der 5.~Zeile dreimal auf und geben Sie auch die Variable
\texttt{Augenzahl3} in der 14.~Zeile nach einem weiteren Beistrich an.
Weiters, nutzen Sie Ihre Lösung für die Aufgabe 4
im 2.~Übungsblatt und erweitern Sie hier die 15.~Zeile, so dass Sie nach dieser Art eine neue Lösungsmethode
erhalten.

\item Ändern Sie Ihr Programm in der 5.~Zeile nach dieser Art:
\begin{lstlisting}[language=Python,firstnumber=5]
Ω = set(itertools.product(Augenzahlen, repeat=3))
\end{lstlisting}
Wann kann diese Schreibweise nützlich sein?

\end{enumerate}
\end{document}
