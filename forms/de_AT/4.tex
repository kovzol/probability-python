\documentclass{article}
\usepackage{listingsutf8,newtxtt,xcolor,graphicx}
\usepackage[hyphens]{url}
\addtolength{\topmargin}{-.875in}
\addtolength{\textheight}{1.75in}
\usepackage{hyperref}
\usepackage{amsmath,amssymb}
\begin{document}
\title{Übungsblatt 4}
\date{}
\maketitle
\lstset{language=Python,
  numbers=left,
  basicstyle=\ttfamily,
  keywordstyle=\bfseries\color{magenta},
  stringstyle=\color{red},
  numberstyle=\color{blue},
  breaklines=true,
  showstringspaces=false,
  backgroundcolor=\color{yellow!30},
  ndkeywordstyle=\color{magenta!70},
  commentstyle=\color{green},
  identifierstyle=\color{black},
  literate=
  {á}{{\'a}}1 {é}{{\'e}}1 {í}{{\'i}}1 {ó}{{\'o}}1 {ú}{{\'u}}1
  {Á}{{\'A}}1 {É}{{\'E}}1 {Í}{{\'I}}1 {Ó}{{\'O}}1 {Ú}{{\'U}}1
  {à}{{\`a}}1 {è}{{\`e}}1 {ì}{{\`i}}1 {ò}{{\`o}}1 {ù}{{\`u}}1
  {À}{{\`A}}1 {È}{{\'E}}1 {Ì}{{\`I}}1 {Ò}{{\`O}}1 {Ù}{{\`U}}1
  {ä}{{\"a}}1 {ë}{{\"e}}1 {ï}{{\"i}}1 {ö}{{\"o}}1 {ü}{{\"u}}1
  {Ä}{{\"A}}1 {Ë}{{\"E}}1 {Ï}{{\"I}}1 {Ö}{{\"O}}1 {Ü}{{\"U}}1
  {â}{{\^a}}1 {ê}{{\^e}}1 {î}{{\^i}}1 {ô}{{\^o}}1 {û}{{\^u}}1
  {Â}{{\^A}}1 {Ê}{{\^E}}1 {Î}{{\^I}}1 {Ô}{{\^O}}1 {Û}{{\^U}}1
  {Ã}{{\~A}}1 {ã}{{\~a}}1 {Õ}{{\~O}}1 {õ}{{\~o}}1
  {œ}{{\oe}}1 {Œ}{{\OE}}1 {æ}{{\ae}}1 {Æ}{{\AE}}1 {ß}{{\ss}}1
  {ű}{{\H{u}}}1 {Ű}{{\H{U}}}1 {ő}{{\H{o}}}1 {Ő}{{\H{O}}}1
  {ç}{{\c c}}1 {Ç}{{\c C}}1 {ø}{{\o}}1 {å}{{\r a}}1 {Å}{{\r A}}1
  {€}{{\euro}}1 {£}{{\pounds}}1 {«}{{\guillemotleft}}1
  {»}{{\guillemotright}}1 {ñ}{{\~n}}1 {Ñ}{{\~N}}1 {¿}{{?`}}1
  {Ω}{{$\Omega$}}1
  {ω}{{$\omega$}}1
  {→}{{$\to$}}1
  {ℝ}{{$\mathbb{R}$}}1,
  columns=fullflexible,
  keepspaces=true
}
\makeatletter
\def\lst@outputspace{{\ifx\lst@bkgcolor\empty\color{white}\else\lst@bkgcolor\fi\lst@visiblespace}}
\makeatother
\thispagestyle{empty}
\begin{enumerate}
\item Betrachten wir folgendes Python3 Programm:
\lstinputlisting[language=Python]{../../progs/de_AT/4.py}

\begin{enumerate}
\item Überprüfen Sie, ob das Programm richtig funktioniert. Speichern Sie als Beweis ein Bildschirmfoto ab.
\item Erklären Sie die Unterschiede in diesem Programm und dem im Übungsblatt 3 (Aufgabe 1).
\end{enumerate}

\item Ändern Sie Ihr Programm in den 7.~und 14.~Zeilen, so dass Sie folgendes Beispiel lösen können:
\begin{quote}
Anna und Bernhard würfeln viermal hintereinander mit einem Würfel. 
Wenn die Augensumme zwischen 7 und 14 ist, gewinnt Anna, sonst Bernhard.
Welcher Spieler wird bei dem Spiel begünstigt?
\end{quote}

\item Ändern Sie die Spielregeln in der obigen Aufgabe, so dass das Spiel gerecht sei.
Überprüfen Sie Ihre Aufgabenstellung mithilfe von einem passenden Python3 Programm.

\item Wir ersetzen die Zeilen 12--15 im obigen Programm mit der Zeile
\begin{lstlisting}[language=Python,numbers=none]
E = [ω for ω in Ω if X(ω) == 10]
\end{lstlisting}
Geben Sie diese Zeile in einer mathematischen Form als eine Mengendefinition an.

\end{enumerate}
\end{document}
