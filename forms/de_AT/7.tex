\documentclass{article}
\usepackage{listingsutf8,newtxtt,xcolor,graphicx}
\usepackage[hyphens]{xurl}
\usepackage{hyperref}
\usepackage{amsmath,amssymb}
\usepackage{geometry}
\addtolength{\topmargin}{-.875in}
\addtolength{\textheight}{1.75in}
\begin{document}
\title{Übungsblatt 7}
\date{}
\maketitle
\lstset{language=Python,
  numbers=left,
  basicstyle=\ttfamily,
  keywordstyle=\bfseries\color{magenta},
  stringstyle=\color{red},
  numberstyle=\color{blue},
  breaklines=true,
  showstringspaces=false,
  backgroundcolor=\color{yellow!30},
  ndkeywordstyle=\color{magenta!70},
  commentstyle=\color{green},
  identifierstyle=\color{black},
  literate=
  {á}{{\'a}}1 {é}{{\'e}}1 {í}{{\'i}}1 {ó}{{\'o}}1 {ú}{{\'u}}1
  {Á}{{\'A}}1 {É}{{\'E}}1 {Í}{{\'I}}1 {Ó}{{\'O}}1 {Ú}{{\'U}}1
  {à}{{\`a}}1 {è}{{\`e}}1 {ì}{{\`i}}1 {ò}{{\`o}}1 {ù}{{\`u}}1
  {À}{{\`A}}1 {È}{{\'E}}1 {Ì}{{\`I}}1 {Ò}{{\`O}}1 {Ù}{{\`U}}1
  {ä}{{\"a}}1 {ë}{{\"e}}1 {ï}{{\"i}}1 {ö}{{\"o}}1 {ü}{{\"u}}1
  {Ä}{{\"A}}1 {Ë}{{\"E}}1 {Ï}{{\"I}}1 {Ö}{{\"O}}1 {Ü}{{\"U}}1
  {â}{{\^a}}1 {ê}{{\^e}}1 {î}{{\^i}}1 {ô}{{\^o}}1 {û}{{\^u}}1
  {Â}{{\^A}}1 {Ê}{{\^E}}1 {Î}{{\^I}}1 {Ô}{{\^O}}1 {Û}{{\^U}}1
  {Ã}{{\~A}}1 {ã}{{\~a}}1 {Õ}{{\~O}}1 {õ}{{\~o}}1
  {œ}{{\oe}}1 {Œ}{{\OE}}1 {æ}{{\ae}}1 {Æ}{{\AE}}1 {ß}{{\ss}}1
  {ű}{{\H{u}}}1 {Ű}{{\H{U}}}1 {ő}{{\H{o}}}1 {Ő}{{\H{O}}}1
  {ç}{{\c c}}1 {Ç}{{\c C}}1 {ø}{{\o}}1 {å}{{\r a}}1 {Å}{{\r A}}1
  {€}{{\euro}}1 {£}{{\pounds}}1 {«}{{\guillemotleft}}1
  {»}{{\guillemotright}}1 {ñ}{{\~n}}1 {Ñ}{{\~N}}1 {¿}{{?`}}1
  {Ω}{{$\Omega$}}1
  {ω}{{$\omega$}}1
  {→}{{$\to$}}1
  {ℝ}{{$\mathbb{R}$}}1,
  columns=fullflexible,
  keepspaces=true
}
\makeatletter
\def\lst@outputspace{{\ifx\lst@bkgcolor\empty\color{white}\else\lst@bkgcolor\fi\lst@visiblespace}}
\makeatother
\thispagestyle{empty}

\noindent Betrachten wir folgendes Programm:
\lstinputlisting[language=Python]{../../progs/de_AT/7a.py}

\begin{enumerate}
\item Welche Resultate werden vom Programm berechnet?
\item Erstellen Sie Liniendiagramme, welche die Resultate grafisch darstellen. (Für jede
Dichtefunktion sollten Sie eine einzige Kurve erstellen und alle Kurven in demselben Graphen
dar\-stellen.) Tipp: Falls Sie die Website \url{cocalc.com} nutzen und ein Sage Worksheet
erstellen, können Sie alle Liniendiagramme automatisch darstellen:
\lstinputlisting[language=Python]{../../progs/de_AT/7b.py}
Eventuell können Sie statt der 10.~Zeile folgenden Befehl nutzen:
\begin{lstlisting}[language=Python,firstnumber=10]
    p.bar(range(1, max_Würfe * 6 + 1), Wahrscheinlichkeiten[1:], alpha=0.5)
\end{lstlisting}
Welchen grafischen Unterschied gibt es zwischen den zwei Darstellungen?
\item Erklären Sie mit möglichst präzisen mathematischen Begriffen, wie die Liniendiagramme interpretiert werden können.
Wichtige Hinweise:
\begin{enumerate}
\item Lesen Sie so viele Eigenschaften ab, wie es möglich ist.
\item Verallgemeinern Sie die Resultate.
\item Nutzen Sie nur solche Begriffe, die in der Schule (Unterstufe oder Oberstufe) erklärt werden können.
(Wir nehmen an, dass folgende Begriffe schon bekannt sind: Folge, Grenz\-wert, Funktion,
Dichtefunktion, Erwartungswert, Varianz, Streuung, Normalverteilung.)
\end{enumerate}

\end{enumerate}
\end{document}
